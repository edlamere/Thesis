%
% Modified by Megan Patnott
% Last Change: Jan 18, 2013
%
%%%%%%%%%%%%%%%%%%%%%%%%%%%%%%%%%%%%%%%%%%%%%%%%%%%%%%%%%%%%%%%%%%%%%%%%
%
% Modified version of the sample_ndthesis.tex
% by Sameer Vijay
% Last Change: Wed Jul 27 2005 14:00 CEST
%
%%%%%%%%%%%%%%%%%%%%%%%%%%%%%%%%%%%%%%%%%%%%%%%%%%%%%%%%%%%%%%%%%%%%%%%%
%
% Sample Notre Dame Thesis/Dissertation
% Using Donald Peterson's ndthesis classfile
%
% Written by Jeff Squyres and Don Peterson
%
% Provided by the Information Technology Committee of
%   the Graduate Student Union
%   http://www.gsu.nd.edu/
%
% Nothing in this document is serious except the format.  :-)
%
%%%%%%%%%%%%%%%%%%%%%%%%%%%%%%%%%%%%%%%%%%%%%%%%%%%%%%%%%%%%%%%%%%%%%%%%
% This is *not* a substitute for the documentation, which is included
% as a pdf file in the standard distribution, and can be obtained from
% the dtx file in the advanced distribution.
%%%%%%%%%%%%%%%%%%%%%%%%%%%%%%%%%%%%%%%%%%%%%%%%%%%%%%%%%%%%%%%%%%%%%%%%
%
% You should *also* have a ND formatting guide to ensure that you have
% all the relevant parts, put the captions in the right place, etc.
% Just because you have this wonderful style classfile doesn't mean
% that it removes *all* the formatting onus from you.  :-)
% Although be warned that the Graduate School has been known to let
% their official formatting guide get out of date. When in doubt,
% the Microsoft Word example seemed to be the only thing kept
% consistently up-to-date in 2013, and is probably the safest thing
% to consult.
%
% You should break all of this stuff up into separate files
% (at the very least, one chapter per file) and use the \include
% command, as has been done here for chapters 1 and 2 and the appendix.
% There is also an \input command, but \include is more commonly used to
% import chapters in books and dissertations. For the differences between these
% two commands, see, e.g., 
% http://web.science.mq.edu.au/~rdale/resources/writingnotes/latexstruct.html
% or http://tex.stackexchange.com/questions/246/when-should-i-use-input-vs-include.
%
% If you compile from the command line, note that you should also have 
% a good Makefile; one that invokes LaTeX as many times as necessary 
% (up to 4) and bibtex if necessary.
%
% If you use an editor that allows you to compile from within the
% program, note that you will need to compile up to four times. Also,
% we recommend that you use pdflatex (sometimes displayed as
% LaTeX => PDF) to compile directly to pdf.
%
% If you have any suggestions, comments, questions, please send e-mail
% to: dteditor@nd.edu
%
%%%%%%%%%%%%%%%%%%%%%%%%%%%%%%%%%%%%%%%%%%%%%%%%%%%%%%%%%%%%%%%%%%%%%%%%

\documentclass[final,numrefs,sort&compress, noinfo]{nddiss2e}
% \documentclass[final,numrefs,sort&compress,twoadvisors]{nddiss2e}
% One of the options draft, review, final must be chosen.
% One of the options textrefs or numrefs should be chosen
% to specify if you want numerical or ``author-date''
% style citations.
% Other available options are:
% 10pt/11pt/12pt (available with draft only)
% twoadvisors
% noinfo (should be used when you compile the final time
%         for formal submission)
% sort (sorts multiple citations in the order that they're
%       listed in the bibliography)
% compress (compresses numerical citations, e.g. [1,2,3]
%           becomes [1-3]; has no effect when used with
%           the textrefs option)
% sort&compress (sorts and compresses numerical citations;
%           is identical to sort when used with textrefs)

\begin{document}

\frontmatter % All the items before the first chapter go in ``frontmatter''

% Your title must be in all caps, and you must do this manually!
% Titles may be 1-4 lines long. If your title is longer than 4 lines, the class file
% may have difficulty formatting the title page.
\title{SYSTEMATIC STUDY OF TECHNETIUM CONTAMINATION \\ THROUGH PROTON BOMBARDMENT OF MOLYBDENUM }
\author{Edward A. Lamere}
\work{Dissertation} % or \work{Thesis}
\degaward{Doctor of Philosophy} % or 
% \degaward{Master of Science \\ in \\ Subject}
\advisor{Manoel Coud{\"e}r}
%\secondadvisor{Gordon Gray} % if you have two advisers are using the option twoadvisors
\department{Physics}

\maketitle
%%%%%%%%%%%%%%%%%%%%%%%%%%%%%%%%%%%%%%%%%%%%%%%%%%%%%%%%%%%%%%%%%%%%%%%%
%
% Front stuff
%
%%%%%%%%%%%%%%%%%%%%%%%%%%%%%%%%%%%%%%%%%%%%%%%%%%%%%%%%%%%%%%%%%%%%%%%%

% You must either set the copyright information or put your work in the public domain.
\copyrightholder{Edeard Lamere} % See template or documentation for
\copyrightyear{2017}           % other copyright options.
\makecopyright

% An abstract is optional for a mster's thesis, and required for a doctoral dissertation.
\begin{abstract}
% ADD MY ABSTRACT HERE!!!
\end{abstract}

% A dedication is optional.
\renewcommand{\dedicationname}{Dedication}
\begin{dedication}
% ADD MY DEDICATION HERE
\end{dedication}

% These are required, and must be in this order.
\tableofcontents
\listoffigures
\listoftables

% A preface is optional.
% \begin{preface}
% ADD A PREFACE???
% \end{preface}

% It's hard to tell from the information available from the Graduate
% School in Spring 2013 whether or not an acknowledgements section is optional.
\begin{acknowledge}
  I would like to acknowledge the invaluable mentorship of Manoel Couder...
\end{acknowledge}

% A symbols section is optional.
\begin{symbols}
  \sym{E}{energy}  
\end{symbols}

\mainmatter
% Place the text body here.
%\include{chapter-one}
%Begin each chapter with \chapter{TITLE}. Chapter titles must be in all caps
%and ensuring that they are is your responsibility.

%
% An unnumbered chapter (features)
%
% \unnumchapter{FEATURES OF FORMATTING IN THIS EXAMPLE FILE}
% The \unnumchapter command allows you to include an unnumbered chapter as part of
% the main text before Chapter 1. It will appear in your table of contents, and you
% should have at most one such chapter (although nothing in the class file will
% prevent you from creating more).

% The usual \cite{} command is also available, and should work as expected.

%
% Outline
%
% %
% Outline
%

\chapter{Outline}
Here is some text.
\section{Introduction}
% And some more\cite{gnus98:_gerry_ganst}.
\subsection{Background}


% % uncomment the following lines,
% if using chapter-wise bibliography
%
% \bibliographystyle{ndnatbib}
% \bibliography{example}


%
% Chapter 1
%

\include{./Chapters/chapter1}


%
% Chapter 2
%

% \include{./Chapters/chapter2}


%
% Appendix (optional)
%

% \appendix

% \include{./Chapters/appendix}


%
% Back stuff
%

% % comment out the following three lines
% if using chapter-wise bibliography

 \backmatter
 \bibliographystyle{abbrvnat} % The standard abbrvnat style should be acceptable. Also provided with both the advanced and standard
 \bibliography{./Chapters/example}       % distributions are nddiss2e and nddiss2enoarticletitles style options.
% If you prefer to manually enter your bibliography, that is fine. Comment out the previous two lines, and enter your bibliography
% as usual. Note that if you choose this route, formatting the bibliography is your responsibility. An example is below, including the
% optional arguments necessary for author-date style citations.
%	\begin{thebibliography}{9}
%		\bibitem[Galmira(1998)]{galmira98:_gnus_milit} G.\ Galmira. Gnus and the military -- a secret conspiracy? \emph{Growing Towards Gnu}, III(7):22--183, September 1998.
%		
%		\bibitem[Ganston and Greenfield(1998)]{gnus98:_gerry_ganst} G.\ Ganston and G.\ Greenfield. \emph{Gnus and You: The Art of Being New}. volume I. Grapping Books, NY, August, 1998.
%		
%		\bibitem[Gloonston(1998)]{gloonston98:_gnuly_discov_gnus} G.\ Gloonston. Newly discovered gnus: The LoG. \emph{Growing Towards Gnu}, II(12):23---57, March 1998.
%		
%		\bibitem[Greenfield(1996)]{greenfield96:_gettin_know_gnu} G.\ Greenfield. \emph{Getting to Know Gnu}. PhD thesis, Geoffrey Garfield School of Gnus, August 1996.
%		
%		\bibitem[van Gairley(2000)]{gairley2000} G.\ van Gairley. Gnu's review. Website, 2000. \url{http://www.gairley.gnu}.
%	\end{thebibliography}

\end{document}

% End of ``example.tex''
